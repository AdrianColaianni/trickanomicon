\documentclass[12pt,letterpaper]{article}
\usepackage[margin=1in]{geometry}
\usepackage[T1]{fontenc}

% Font w/ xelatex
\ifxetex
	\usepackage{fontspec}
	\setsansfont{Jost.ttf}[
	BoldFont = Jost-Bold.ttf,
	ItalicFont = Jost-Italic.ttf,
	BoldItalicFont = Jost-BoldItalic.ttf]
	\renewcommand{\familydefault}{\sfdefault}
	\setmonofont{FiraCode Nerd Font}
\else
	\renewcommand{\familydefault}{\sfdefault}
\fi

% Quotes
\usepackage[american]{babel}

% Rose Pine
\usepackage{xcolor}
\definecolor{fg}{HTML}{e0def4}
\definecolor{bg}{HTML}{232136}
\definecolor{c0}{HTML}{393552}
\definecolor{c1}{HTML}{eb6f92}
\definecolor{c2}{HTML}{9ccfd8}
\definecolor{c3}{HTML}{f6c177}
\definecolor{c4}{HTML}{3e8fb0}
\definecolor{c5}{HTML}{c4a7e7}
\definecolor{c6}{HTML}{ea9a97}
\definecolor{c7}{HTML}{e0def4}
\definecolor{c8}{HTML}{6e6a86}
\pagecolor{bg}
\color{fg}

% Images
% \usepackage{graphicx}
% \usepackage{wrapfig}
% \usepackage{float}

% \usepackage{pgfplots}
% \usepackage{multicol}
% \usepackage{lipsum}

\def\code#1{\textcolor{c2}{\texttt{#1}}}
\def\bf#1{\textbf{#1}}
\def\ul#1{\underline{#1}}
\def\it#1{\textit{#1}}

% Links
\usepackage{hyperref}
\hypersetup{
    colorlinks=true,
    linkcolor=c4,
    filecolor=magenta,
    urlcolor=blue,
}

\title{\huge{\textit{Trickanomicon}}}
\author{Clemson CCDC Team}
\date{2024}

\begin{document}

\maketitle

\pagebreak

\tableofcontents

\pagebreak

\section{Linux}

\subsection{30 Minute Plan}

\begin{enumerate}
	\item Rotate all ssh keys
		\begin{enumerate}
			\item Generate a new key with \code{ssh-keygen -t ed25519}
			\item Distribute to all teammates via private Discord channels
			\item Deploy to boxes
			\item Remove all other authorized keys with \code{find / -name authorized\_keys}
		\end{enumerate}
	\item Change all ssh ports
	\item Check user accounts in \code{/etc/passwd}
\end{enumerate}

\subsection{Monitoring}

\begin{enumerate}
	\item View all network connections \code{ss -tunlp}
	\item View listening programs with \code{ss -lp}
	\item View only connections with \code{ss -tu}
	\item View active processes \code{ps -e}
	\item Continuously see processes with \code{top}
		\begin{enumerate}
			\item Sort by different categories with \code{<} and \code{>}
			\item Tree view with \code{V}
		\end{enumerate}
	\item Continuously see network connections with \code{watch ss -tu}
\end{enumerate}

\subsection{System Utilities}

\begin{enumerate}
	\item Start and stop processes with \code{systemctl}
	\begin{enumerate}
		\item Start service with \code{\bf{systemctl start} [\ul{UNIT}]}
		\item Stop service with \code{\bf{systemctl stop} [\ul{UNIT}]}
		\item Restart service with \code{\bf{systemctl restart} [\ul{UNIT}]}
		\item Enable service with \code{\bf{systemctl enable} [\ul{UNIT}]}
	\end{enumerate}
	\item Permit and allow network connections with \code{iptables}
	\begin{enumerate}
		\item Default deny with \code{iptables -P INPUT DROP}
		\item Allow port access (must choose tcp or udp) \\
			\code{\bf{iptables -A INPUT -p tcp|udp --dport} \ul{port} \bf{-j ACCEPT}}
		\item Allow all from interface \\
			\code{\bf{iptables -A INPUT -i} \ul{interface} \bf{-j ACCEPT}}
	\end{enumerate}
\end{enumerate}

\subsection{Hunting}

\begin{enumerate}
	\item List all files with creation date, most recent first: \code{find /usr /bin /etc /var -type f -exec stat -c "\%W \%n" {} + | sort -r > files}
\end{enumerate}

\pagebreak

\section{Windows}

\subsection{30 Minute Plan}

\begin{enumerate}
	\item (Recommended) Install helpful system utilities.
		\begin{enumerate}
			\item Set up scoop package manager. \\
				\code{Set-ExecutionPolicy -ExecutionPolicy RemoteSigned -Scope CurrentUser; Invoke-RestMethod -Uri https://get.scoop.sh | Invoke-Expression}
			\item Add the required repositories. \\
				\code{scoop bucket add sysinternals; scoop bucket add extras}
			\item Install Sysinternals Suite and Nmap. \\
				\code{scoop install sysinternals-suite nmap extras/vcredist}
		\end{enumerate}
	\item Change all authorized user account passwords except for \bf{seccdc\_black}. \\
		\bf{You can use the following one-liner to help expedite the process:}
		\begin{verbatim}
		get-content "users.txt" | foreach {
			$secret = -join (([char]'A'..[char]'Z' + [char]'#'..[char]'&') |
			get-random -Count 24 | % {[char]$_});
			net user $_ $secret;
			add-content "output.csv" "hostname,$_,$secret"
		}
		\end{verbatim}
	\bf{NOTE: If in an AD environment, run above twice, once with /domain.}
	\item Disable unauthorized user accounts on the system. \\
		\bf{You can use the following one-liners to help expedite the process:}
		\begin{enumerate}
			\item Audit accounts on system: \\
			\begin{verbatim}
			get-localuser | foreach {
				$expected = get-content "users.txt";
				if ($_.Name -notin $expected) {
					echo $_.Name; add-content "unexpected.txt" $_.Name
				}
			}
			\end{verbatim}
			\item Disable any unauthorized accounts:
				\begin{verbatim}
				get-content "unexpected.txt" | foreach {net user $_ /active: no}
				\end{verbatim}
		\end{enumerate}
		\bf{NOTE: If in an AD environment, run above with get-aduser and /domain.}
	\item Perform an Nmap scan on machine and note which ports should be accessible.
	\item Once scan completes, configure Firewall.
		\begin{enumerate}
			\item Export current firewall policy.
			\item Disable firewall.
			\item Flush inbound/outbound rules.
			\item Add a rule for RDP.
			\item Configure additional rules as needed for scoring/outbound internet.
			\item Re-enable firewall.
		\end{enumerate}
\end{enumerate}

\subsection{Hardening}

\begin{enumerate}
	\item Configure NLA for RDP.
	\item Check User Rights Assignment Settings.
	\item Service Management:
	\begin{enumerate}
		\item Disable Print Spooler. \\
			\code{Set-Service -Name "Spooler" -Status stopped -StartupType disabled}
		\item Disable WinRM. \\
			\code{Disable-PSRemoting -Force}; \\
			\code{Set-Service -Name "WinRM" -Status stopped -StartupType disabled}
		\item Configure SMB:
		\begin{enumerate}
			\item If SMB is unneeded (i.e. not in an AD setting), disable it entirely. \\
			\code{Set-Service -Name "LanmanServer" -Status stopped -StartupType disabled}
			\item If SMB is needed, disable SMBv1. \\
			\code{Disable-WindowsOptionalFeature -Online -FeatureName smb1protocol} 
		\end{enumerate}
		\item Ensure SSH has not been compromised.
		\begin{enumerate}
			\item Find and remove authorized keys. \\
				\code{dir C:\textbackslash{} -Recurse | findstr "authorized\_keys"}
			\item Check config at \%programdata\%\textbackslash{}ssh\textbackslash{}sshd\_config file for \href{https://learn.microsoft.com/en-us/windows-server/administration/openssh/openssh_server_configuration}{misconfigurations}. 
		\end{enumerate}
	\end{enumerate}
\end{enumerate}

\pagebreak

\subsection{Monitoring}

\begin{enumerate}
	\item View all network connections with \code{netstat -aonb}. For a live view, use TCPView.
	\item View running processes in the details pane of Task Manager or via Process Explorer.
	\item View all connected Named Pipes / SMB Shares with \code{net use}.
	\item View all connected sessions with \code{qwinsta}.
	\item To get an insight into Powershell activity, enable Powershell Block Logging/Transcription. \\
		\code{Administrative Templates > Windows Components > Windows Powershell}
	\item For a live overview of system activity, configure Sysmon.
	\begin{enumerate}
		\item To install Sysmon, run \code{sysmon -i <PATH TO CONFIG FILE>}.
		\item Logs are sent to Applications and Services Logs > Microsoft > Windows > Sysmon.
		\item If you want to update your config file, run \code{sysmon -c <PATH TO NEW CONFIG>}. 
	\end{enumerate}
\end{enumerate}

\subsection{Hunting}

\begin{enumerate}
	\item Install Malwarebytes and run a system scan. It can be installed silently with: \\
		\code{curl https://downloads.malwarebytes.com/file/mb-windows -OutFile "MBSetup.exe"; .\textbackslash{}MBSetup.exe /VERYSILENT /NORESTART}
	\item You can scan the system for unsigned dlls with \code{listdlls -u}.
	\item AccessEnum can be used to search for misconfigured ACLs. Check sensitive registry keys/directories.
	\item The Autoruns utility can be used to find potential persistence mechanisms. The Task Scheduler should also be checked.
	\item You can use BLUESPAWN to aid in hunting. It is a complete EDR solution built for CCDC.
	\begin{enumerate}
		\item It can be obtained from \href{https://github.com/ION28/BLUESPAWN/releases/download/v0.5.1-alpha/BLUESPAWN-client-x64.exe}{here}.
		\item Basic usage is as follows:
		\begin{enumerate}
			\item BLUESPAWN-client-x64.exe --mitigate compares the system to a secure baseline.
			\item BLUESPAWN-client-x64.exe --hunt does a single-pass scan for malicious activity.
			\item BLUESPAWN-client-x64.exe --monitor is like hunt but alerts on new activity.
		\end{enumerate}
	\end{enumerate}
\end{enumerate}

\subsection{Response}

\begin{enumerate}
	\item Kill a connected session with \code{rwinsta <SESSION ID>}.
	\item Kill a process in Task Manager/Process Explorer or with \code{taskkill /f /pid <PID>}.
\end{enumerate}

\pagebreak

\subsection{Installing the ELK Stack}

\begin{enumerate}
	\item Install the prerequisites	using manually or using scoop (see below). \\
	\ul{If using scoop, apps are installed in \$HOME/scoop/apps/<APPNAME>/current.}
		\begin{enumerate}
			\item Add the `extras` repository with \code{scoop bucket add extras}.
			\item Add the `java' repository with \code{scoop bucket add java}.
			\item Install elasticsearch and kibana with \code{scoop install elasticsearch kibana openjdk}.
			\item Notepad++ (notepadplusplus) is recommended for editing the yml files.
		\end{enumerate}
	\item Add the following lines to \code{<ELASTICSEARCH DIR>/config/elasticsearch.yml}:
		\begin{verbatim}
			xpack.security.enabled: true
			xpack.security.enrollment.enabled: true
			cluster.initial_master_nodes: ["elk"]
			http.host: [\_local\_, \_site\_]		
		\end{verbatim}
	\item Start elasticsearch in a Powershell prompt. It will take a moment to initialize.
	\item Reset passwords for the kibana\_system and elastic accounts.
		\begin{enumerate}
			\item \code{<ELASTICSEARCH DIR>/bin/elasticsearch-reset-password -u kibana-system}
			\item \code{<ELASTICSEARCH DIR>/bin/elasticsearch-reset-password -u elastic}
		\end{enumerate}
	\bf{SAVE THESE PASSWORDS SOMEWHERE.}
	\item Add the following lines to \code{<KIBANA DIR>/config/kibana.yml}:
		\begin{verbatim}
			server.host: "0.0.0.0"
			server.publicBaseUrl: "http://localhost"
			elasticsearch.hosts: ["http://localhost:9200"]
			elasticsearch.username: "kibana\_system"
			elasticsearch.password: "<GENERATED KIBANA-SYSTEM PASSWORD>"
			elasticsearch.ssl.verificationMode: none
		\end{verbatim}
	\item Stop the elasticsearch executable and install it as a service with: \\
		\code{<ELASTICSEARCH DIR>/bin/elasticsearch-service.bat install} 
	\item Start the elasticsearch service with \code{start-service elasticsearch-service-x64}
	\item Start kibana in a Powershell prompt. It will take a moment to initialize.
	\item You may now log into kibana at localhost:5601 with the elastic account.
\end{enumerate}

\pagebreak

\subsection{Configuring Winlogbeat}

\begin{enumerate}
	\item You can obtain Winlogbeat from \href{https://artifacts.elastic.co/downloads/beats/winlogbeat/winlogbeat-8.12.0-windows-x86_64.zip}{here}.
	\item Extract the Winlogbeat directory to C:\textbackslash{}Program Files.
	\item Edit the winlogbeat.yml file as needed for environment and log forwarding.
	\item Install Winlogbeat as a service with \code{.\textbackslash{}install-service-winlogbeat.ps1}
	\item Load the prebuilt Winlogbeat assets with \code{.\textbackslash{}winlogbeat.exe setup -e}.
	\item Start the Winlogbeat service with \code{start-service winlogbeat}
\end{enumerate}

\pagebreak

\section{Service Configuration}

\subsection{MySQL}

\begin{enumerate}
	\item Change passwords for any non-scored users. \\
		\code{SELECT user FROM mysql.user;} \\
		\code{ALTER USER '<USERNAME>'@localhost IDENTIFIED BY '<NEW PASSWORD>';}
	\item Prepare a database backup and store somewhere safe on the system.
	\begin{enumerate}
		\item Backup: \code{mysqldump -u <USERNAME> -p > <PLACE TO STORE BACKUP>}
		\item Restore: \code{mysql -u <USERNAME> -p < <BACKUP LOCATION>} \\
	\end{enumerate}
\end{enumerate}

\subsection{FTP}

\begin{enumerate}
	\item
\end{enumerate}

\end{document}