\documentclass[12pt,letterpaper]{article}
\usepackage[margin=1in]{geometry}
\usepackage[T1]{fontenc}

% Font w/ xelatex
\ifxetex
	\usepackage{fontspec}
	\setsansfont{Jost.ttf}[
	BoldFont = Jost-Bold.ttf ,
	ItalicFont = Jost-Italic.ttf ,
	BoldItalicFont = Jost-BoldItalic.ttf]
	\renewcommand{\familydefault}{\sfdefault}
	\setmonofont{FiraCode Nerd Font}
\else
	\renewcommand{\familydefault}{\sfdefault}
\fi

% Quotes
\usepackage[american]{babel}

% Rose Pine
\usepackage{xcolor}
% \def\light{1}
\ifx\light\defined
	\definecolor{base}{HTML}{1f1d2e}
	\definecolor{muted}{HTML}{6e6a86}
	\definecolor{subtle}{HTML}{908caa}
	\definecolor{text}{HTML}{e0def4}
	\definecolor{love}{HTML}{eb6f92}
	\definecolor{gold}{HTML}{f6c177}
	\definecolor{rose}{HTML}{ea9a97}
	\definecolor{pine}{HTML}{3e8fb0}
	\definecolor{foam}{HTML}{9ccfd8}
	\definecolor{iris}{HTML}{c4a7e7}
	\pagecolor{base}
	\color{text}
\else
	\definecolor{base}{HTML}{faf4ed}
	\definecolor{muted}{HTML}{9893a5}
	\definecolor{subtle}{HTML}{797593}
	\definecolor{text}{HTML}{575279}
	\definecolor{love}{HTML}{b4637a}
	\definecolor{gold}{HTML}{ea9d34}
	\definecolor{rose}{HTML}{d7827e}
	\definecolor{pine}{HTML}{286983}
	\definecolor{foam}{HTML}{56949f}
	\definecolor{iris}{HTML}{907aa9}
	\pagecolor{base}
	\color{text}
\fi

% Images
% \usepackage{graphicx}
% \usepackage{wrapfig}
% \usepackage{float}

% \usepackage{pgfplots}
% \usepackage{multicol}
% \usepackage{lipsum}

% Set list numbering
\renewcommand{\labelenumi}{{\color{iris}\arabic{enumi}}}
\renewcommand{\labelenumii}{{\color{iris}\arabic{enumi}.\arabic{enumii}}}
\renewcommand{\labelenumiii}{{\color{iris}\arabic{enumi}.\arabic{enumii}.\arabic{enumiii}}}
\renewcommand{\labelenumiv}{{\color{iris}\arabic{enumi}.\arabic{enumii}.\arabic{enumiii}.\arabic{enumiv}}}

% Set heading colors
\usepackage{sectsty}
\sectionfont{\color{foam}}
\subsectionfont{\color{rose}}

\def\code#1{\textcolor{iris}{\texttt{#1}}}
\def\bf#1{\textbf{#1}}
\def\ul#1{\underline{#1}}
\def\it#1{\textit{#1}}

% Links
\usepackage{hyperref}
\hypersetup{
    colorlinks=true,
    linkcolor=pine,
    filecolor=magenta,
    urlcolor=pine,
}

\title{\color{iris}\huge{\bf{\it{Trickanomicon}}}}
\author{\color{muted}Clemson CCDC Team}
\date{\color{subtle}2024}

\begin{document}

\maketitle

\pagebreak

\tableofcontents

\pagebreak

\section{Competition Information}

\subsection{Pre-Competition Checklist/General Knowledge Requirements}

\begin{enumerate}
	\item NMAP installed on your local machine
	\item Minimal understanding of bash
\end{enumerate}

\subsection{High-Level First 30 Minutes Plan}

\begin{enumerate}
	\item Lead captain will do a ping sweep scan to see what's on the network + an rdp port scan to figure out which machines are windows

	\item Machines get assigned to people (and this will be written down on a whiteboard that we can all see)

	\item Each person will nmap scan the machine they are assigned and give results in the appropriate channel
		\begin{enumerate} \label{nmap}
		\item Do a quick scan to find all open ports \\
			\code{\bf{nmap -p- -T5} \ul{<target ip>}}
		\item After getting all open ports, get service and OS information with specific ports \\
			\code{\bf{nmap -A -p port1,port2,port3,...} \ul{<target ip>}}
	\end{enumerate}

	\item Each person will log into their machine and do user management stuff

	\item Each person will do system hardening + firewalls (or let firewall guru do their thing) for their machine

	\item Each person will monitor their machine for activity from there on out unless they're helping with an inject
\end{enumerate}

\pagebreak

\section{Linux}

\bf{Do not touch the} \code{seccdc\_black} \bf{account}

\subsection{30 Minute Plan}

\begin{enumerate}
	\item \hyperref[nmap]{NMAP Scan Machine} in background
	\item Rotate all ssh keys
		\begin{enumerate}
			\item Populate the machines with the ssh key from lead linux captain
			\item Deploy with \code{\bf{ssh-copy-id -i} \ul{identity\_file} [\ul{user}@\ul{hostname}]}
			\item Ensure there is only one entry (one line) in \code{/home/<user>/.ssh/authorized\_keys}. If there are more than one, remove all lines EXCEPT the very last one. After saving, make sure you can still SSH by trying on a NEW terminal.
			\item Remove all other authorized keys files with \code{find / -name authorized\_keys}
		\end{enumerate}
	\item Check accounts and reset password \bf{ASAP} using appropriate one liners from \hyperref[subsec:dmagic]{Duncan's Magic}
	\item Lock unnecessary accounts with \code{\bf{usermod -L} \ul{LOGIN}} and if nothing goes red, delete account with \code{\bf{userdel} \ul{LOGIN}}. Or use appropriate one liners from \hyperref[subsec:dmagic]{Duncan's Magic} to lock accounts
		\begin{enumerate}
			\item \bf{NOTE}: user home directories were intentionally not deleted with the \code{userdel} command with the idea of possible needing that data for future injects (you never know). \\ \\
				If you absolutely need to remove extraneous user home directories, seek approval from the team captain before proceeding with the command \code{userdel -r \ul{LOGIN}}
		\end{enumerate}
	\item Find listening services with \code{ss -lp} and investigate strange ones
\end{enumerate}

\subsection{Monitoring}

\begin{enumerate}
	\item View all network connections \code{ss -tunlp}
	\item View listening programs with \code{ss -lp}
	\item View only connections with \code{ss -tu}
	\item View active processes \code{ps -e}
	\item Continuously see processes with \code{top}
		\begin{enumerate}
			\item Sort by different categories with \code{<} and \code{>}
			\item Tree view with \code{V}
		\end{enumerate}
	\item Watch \bf{all} network traffic with the following (this is a fire hose)
		\begin{enumerate}
			\item Record traffic with \code{iptables -I INPUT -j LOG}
			\item Watch input logs with \code{journalctl -kf --grep="IN=.*"}
			\item Watch output logs with \code{journalctl -kf --grep="OUT=.*"}
		\end{enumerate}
	\item Watch dbus with \code{dbus -w}
\end{enumerate}

\subsection{System Utilities}

\begin{enumerate}
	\item Start and stop processes with \code{systemctl}
		\begin{enumerate}
			\item Start service with \code{\bf{systemctl start} [\ul{UNIT}]}
			\item Stop service with \code{\bf{systemctl stop} [\ul{UNIT}]}
			\item Restart service with \code{\bf{systemctl restart} [\ul{UNIT}]}
			\item Enable service with \code{\bf{systemctl enable} [\ul{UNIT}]}
		\end{enumerate}
	\item Permit and allow network connections with \code{iptables}
		\begin{enumerate}
			\item Default deny with \code{iptables -P INPUT DROP}
			\item Allow port access (must choose tcp or udp) \\
				\code{\bf{iptables -A INPUT -p tcp|udp --dport} \ul{port} \bf{-j ACCEPT}}
			\item Allow all from interface \\
				\code{\bf{iptables -A INPUT -i} \ul{interface} \bf{-j ACCEPT}}
		\end{enumerate}
	\item Schedule tasks with cron
		\begin{enumerate}
			\item View with \code{crontab -e}
			\item Obliterate user's crontab with \code{crontab -r}
			\item See \hyperref[subsec:dmagic]{Duncan's Magic} for a one liner to remove crontabs from a list of users. Don't forget to remove \code{sudo}'s crontab(s) too!
		\end{enumerate}
	\item View networking information with \code{ip}
		\begin{enumerate}
			\item View network interfaces with \code{ip l}
				\begin{enumerate}
					\item Interfaces are prefixed with an \code{en} to signify ethernet, \code{wl} to signify wireless, and \code{v} to signify a virtual link
					\item Virtual links should be investigated, as they are commonly used for VPNs, docker, and nonsense
					\item Virtual servers will still show their main link to the host as ethernet
				\end{enumerate}
			\item View IP Addresses with \code{ip a} and take note of additional addresses
		\end{enumerate}
\end{enumerate}

\subsection{Configurations}

\begin{enumerate}
	\item SSH Daemon configs are in \code{/etc/ssh/sshd\_config}
		\begin{enumerate}
			\item \code{PermitRootLogin} should be set to \code{prohibit-password}
			\item \code{UsePAM} should be set to \code{no}
			\item \code{PasswordAuthentication} should be set to \code{yes}
			\item \code{PermitEmptyPasswords} should be set to \code{no}
		\end{enumerate}
	\item File system configs are in \code{/etc/fstab}
		\begin{enumerate}
			\item Network File Shares are in the form \\
				\code{/srv/home    hostname1(rw,sync,no\_subtree\_check)} \\
				These allow sharing file systems over the network.  They must be reviewed to ensure we are not sharing information with attackers
		\end{enumerate}
\end{enumerate}

\subsection{Hunting}

\begin{enumerate}
	\item List all files with creation date, most recent first: \code{find /usr /bin /etc \textbackslash \\
		/var -type f -exec stat -c "\%W \%n" \{\} + | sort -r > files}
	\item List all files created after set date, most recent first: \code{find /usr /bin /etc \textbackslash \\
		/var -type f -newermt \ul{YYYY-MM-DD} -exec stat -c "\%W \%n" \{\} + | sort -r > files}
\end{enumerate}

\subsection{Duncan's Magic}
\label{subsec:dmagic}

\begin{enumerate}
	\item Remove users listed in \bf{./disable.txt} \\
		\code{while read user;do sudo usermod -L \$user;done<disable.txt}
	\item Generate new passwords for every user in \bf{./users.txt}. Output format and filename as specified by SECCDC-2024 password reset guidelines. Ensure SERVICE is changed to match the corresponding service the passwords are being reset for. \\
		\code{t='25'; s='SERVICE'; while read u; do u=`echo \$u | tr -d '[:space:]'`; \textbackslash\\
		p=`tr -dc 'A-Za-z0-9!@\#\$\%' </dev/urandom | head -c 24; echo`; \textbackslash\\
		echo "\$s,\$u,\$p" >{}> "Team"\$t"\_"\$s"\_PWD.csv"; done < users.txt}
	\item Actually reset passwords using the generated list from the command immediately preceeding this \\
		\code{awk -F, '\{ print \$2":"\$3 \}' ./<generated\_file.csv> | sudo chpasswd}
	\item Find users not in \bf{./known\_users.txt} \\
		\code{awk -F: '\{if(\$3>{}=1000\&\&\$3<60000)\{print \$1\}\}' /etc/passwd|sort|comm -13 \textbackslash \\
		known\_users.txt -> extra\_users.txt}
	\item Remove crontabs from list of users in \bf{./users.txt} \\
		\code{while read u; do sudo crontab -u \$u -r; done < users.txt}
\end{enumerate}

\subsection{Hardening}

You should remount /tmp and /var/tmp to be non-executable.
\begin{enumerate}
	\item If /tmp and /var/tmp are already mounted, you can simply do: \code{ \\
		mount -o remount,nodev,nosuid,noexec /tmp \\
		mount -o remount,nodev,nosuid,noexec /var/tmp }
	\item If they haven't already been mounted, edit /etc/fstab to include: \code{ \\
		tmpfs /tmp tmpfs defaults,noatime,nosuid,nodev,noexec,mode=1777 0 0 \\
		tmpfs /var/tmp tmpfs defaults,noatime,nosuid,nodev,noexec,mode=1777 0 0 }
	\item Once /etc/fstab has been updated, you can run: \code{ \\
		mount -o nodev,nosuid,noexec /tmp \\
		mount -o nodev,nosuid,noexec /var/tmp }
\end{enumerate}

\subsection{Logging with auditd}

Setup auditd for logging purposes
\begin{enumerate}
	\item Make sure auditd daemon is enabled and running
		\begin{enumerate}
			\item \code{sudo systemctl enable --now auditd}
		\end{enumerate}
	\item After auditd has been started, add rules to \code{/etc/audit/rules.d/audit.rules}. You will need \code{sudo} permissions to edit this file.
		\begin{enumerate}
			\item Visit \href{https://github.com/Neo23x0/auditd/blob/master/audit.rules}{this masterful auditd github repo} for a wide variety of auditd rules to copy / imitate.
		\end{enumerate}
	\item Restart auditd to apply these new rules
		\begin{enumerate}
			\item \code{sudo systemctl restart auditd}
		\end{enumerate}
\end{enumerate}

Looking at auditd alerts
\begin{enumerate}
	\item Logs are stored in \code{/var/log/audit/audit.log}
		\begin{enumerate}
			\item You can use \code{ausearch} to query this log
			\item \code{aureport} can also be used to generate a list of events
		\end{enumerate}
	\item Important takeaways when analyzing auditd logs
		\begin{enumerate}
			\item euid = effective user id. Pay attention if EUID field is 0 as this means a file or program was run as root
			\item exe field indicates which command was run (if one was run at all)
			\item key field stores the name of the alert that was triggered
			\item pid field stores the process id
		\end{enumerate}
	\item Utilize the various fields, timestamps, and \code{ausearch} and \code{aureport} tools to observe, report, and take action on suspicious activity
\end{enumerate}

\subsection{Service Management}

Unfortunately, this Linux specific service management section is empty at this time as I (Duncan) am tired and want to go to sleep before the competition. \\
Use your best judgement and Google the absolute life out of your fingers. \\
As always, if you get stuck, please ask for help

\pagebreak

\section{Windows}

\bf{Do not touch the} \code{seccdc\_black} \bf{account.}

\subsection{30 Minute Plan}

\begin{enumerate}
	\item Perform an Nmap scan on machine and note which ports should be accessible.
	\item Run/Download the following: \\
		\code{[Net.ServicePointManager]::SecurityProtocol = [Net.SecurityProtocolType]::Tls12}
		\begin{enumerate}
			\item \href{https://download.sysinternals.com/files/SysinternalsSuite.zip}{Sysinternals Suite}, \href{https://raw.githubusercontent.com/D42H5/cyber\_comp\_resources/main/sysmonconfig-export-modified-2-2-24.xml}{Sysmon Config}, \href{https://www.nirsoft.net/utils/fulleventlogview-x64.zip}{EventLogViewer}
			\item \href{https://downloads.malwarebytes.com/file/mb-windows}{Malwarebytes} and \href{https://github.com/ION28/BLUESPAWN/releases/download/v0.5.1-alpha/BLUESPAWN-client-x64.exe}{BLUESPAWN}
		\end{enumerate}
	\item Change authorized account passwords \bf{including your own} with \hyperref[subsec:woliner]{one-liner}.
	\item Disable unauthorized user accounts \bf{except your own and seccdc\_black} with \hyperref[subsec:woliner]{one-liner}.
	\item Find and remove authorized SSH keys.
		\code{dir C:\textbackslash{} -Recurse | findstr "authorized\_keys"}
	\item Once Nmap scan completes, configure Firewall. \\
	\bf{NOTE: Rules SHOULD specify applications AND source/destination IPs. \\Do this via the GUI after rules are made.}	
		\begin{enumerate}
			\item Export current firewall policy. \\
				\code{netsh advfirewall export "C:\textbackslash{}rules.wfw"}
			\item Disable firewall. \\
				\code{netsh advfirewall set allprofiles state off}
			\item Flush inbound/outbound rules. \\
				\code{Remove-NetFirewallRule}
			\item Allow RDP (svchost.exe 3389 TCP), SSH (sshd.exe 22 TCP), and scored service. \\
				\code{\$port = <PORT>; New-NetFirewallRule -DisplayName "Inbound \$port" ` \\
				-Direction Inbound -LocalPort \$port -Protocol TCP ` \\
				-Action Allow -Program "Path\textbackslash{}To\textbackslash{}Executable"}
			\item Configure outbound rules as needed (DNS: 53 TCP/UDP, HTTP: 80,443 TCP). \\
				\code{\$port = <PORT>; New-NetFirewallRule -DisplayName "Outbound \$port" ` \\
				-Direction Outbound -RemotePort \$port -Protocol TCP ` \\
				-Action Allow -Program "Path\textbackslash{}To\textbackslash{}Executable"}
			\item Re-enable firewall to block inbound and outbound (allow outbound on AD). \\
				\code{netsh advfirewall set allprofiles firewallpolicy blockinbound blockoutbound}
				\code{netsh advfirewall set allprofiles state on}
		\end{enumerate}
	\bf{NOTE: If you are in an AD environment, see \hyperref[subsec:adcon]{Active Directory Considerations}.}
\end{enumerate}

\pagebreak

\subsection{One-Liners}
\label{subsec:woliner}

\begin{enumerate}
	\item Install Scoop Package Manager (Recommended): \\
		{ \color{iris} \begin{verbatim}
		Set-ExecutionPolicy -ExecutionPolicy RemoteSigned -Scope CurrentUser;
		Invoke-RestMethod -Uri https://get.scoop.sh | Invoke-Expression
		\end{verbatim} }	
	\item Generate CSV file with Passwords \bf{(NOT NECESSARY FOR QUALIFIERS)}:
		{ \color{iris} \begin{verbatim}
		get-content "users.txt" | foreach {
			$secret = -join (([char]'A'..[char]'Z' + [char]'#'..[char]'&') |
			get-random -Count 24 | % {[char]$\_});
			add-content "Team25\_$(hostname)-SSH\_PWD.csv" "hostname,$\_,$secret"
		}
		\end{verbatim} }
	\item Reset Passwords based on generated password CSV file:
		{ \color{iris} \begin{verbatim}
		import-csv "Team25\_$(hostname)-SSH\_PWD.csv" -Header "host","user","pass" |
		foreach {net user $\_.user $\_.pass};
		del "Team25\_$(hostname)-SSH\_PWD.csv"
		\end{verbatim} }
	\item Audit accounts on system based on list of expected users:			
		{ \color{iris} \begin{verbatim}
			$expected = get-content "users.txt";
			$expected += $env:Username, "seccdc\_black"
			get-localuser | foreach {
				if ($\_.Name -notin $expected) {
					echo $\_.Name; add-content "unexpected.txt" $\_.Name
				}
			}
		\end{verbatim} }
	\item Disable unauthorized accounts:
		{ \color{iris} \begin{verbatim}
		get-content "unexpected.txt" | foreach {net user $\_ /active:no}
		\end{verbatim} }
\end{enumerate}

\pagebreak

\subsection{Hardening}

\begin{enumerate}
	\item Configure NLA for RDP (can also be done through Group Policy).
	\item Service Management:
	\begin{enumerate}
		\item Look at running services and see if any look malicious. Services can be deleted from:
			\code{HKLM\textbackslash{}SYSTEM\textbackslash{}CurrentControlSet\textbackslash{}Services}
		\item Disable Print Spooler. \\
			\code{Set-Service -Name "Spooler" -Status stopped -StartupType disabled}
		\item Disable WinRM. \\
			\code{Disable-PSRemoting -Force}; \\
			\code{Set-Service -Name "WinRM" -Status stopped -StartupType disabled}
		\item Configure SMB:
		\begin{enumerate}
			\item If SMB is unneeded (i.e. not in an AD setting), disable it entirely. \\
				\code{Set-Service -Name "LanmanServer" -Status stopped -StartupType disabled}
			\item If SMB is needed, disable SMBv1. \\
				\code{Disable-WindowsOptionalFeature -Online -FeatureName smb1protocol}
		\end{enumerate}
		\item Harden the scored service for your machine according to the \hyperref[sec:services]{documentation below}.
	\end{enumerate}
	\item Group Policy:
	\begin{enumerate}
		\item Check User Rights Assignment
	\end{enumerate}

\end{enumerate}

\subsection{Monitoring}

\begin{enumerate}
	\item View all network connections with \code{netstat -aonb}. For a live view, use TCPView.
	\item View running processes in the details pane of Task Manager or via Process Explorer.
	\item View all shares with \code{net share} and connected Named Pipes / Shares with \code{net use}.
	\item View all connected sessions with \code{qwinsta}.
	\item To get an insight into Powershell activity, enable Powershell Block Logging. \\
		\code{Administrative Templates > Windows Components > Windows Powershell}
	\item For an overview of system activity, configure Sysmon with \href{https://raw.githubusercontent.com/D42H5/cyber\_comp\_resources/main/sysmonconfig-export-modified-2-2-24.xml}{this config}.
	\begin{enumerate}
		\item To install Sysmon, run \code{sysmon -i <PATH TO CONFIG FILE>}.
		\item Logs are sent to Applications and Services Logs > Microsoft > Windows > Sysmon.
		\item If you want to update your config file, run \code{sysmon -c <PATH TO NEW CONFIG>}.
	\end{enumerate}
	\item You can view system events with the Event Viewer or Nirsoft's \href{https://www.nirsoft.net/utils/fulleventlogview-x64.zip}{EventLogViewer}.
	\begin{enumerate}
		\item You can filter down events to Sysmon*, Powershell*, and Security.
		\item Configure additional auditing as needed.
	\end{enumerate}
\end{enumerate}

\subsection{Response}

\begin{enumerate}
	\item Kill a connected session with \code{rwinsta <SESSION ID>}.
	\item Kill a process in Task Manager/Process Explorer or with \code{taskkill /f /pid <PID>}.
\end{enumerate}

\subsection{Hunting}

\begin{enumerate}
	\item Install \href{https://downloads.malwarebytes.com/file/mb-windows}{Malwarebytes} and run a system scan. It can be installed silently with: \\
		\code{.\textbackslash{}MBSetup.exe /VERYSILENT /NORESTART}
	\item You can scan the system for unsigned dlls with \code{listdlls -u}.
	\item AccessEnum can be used to search for misconfigured ACLs. Check sensitive registry keys/directories. \\
		\code{C:\textbackslash{}Windows\textbackslash{}System32} \\
		\code{HKLM\textbackslash{}SYSTEM\textbackslash{}CurrentControlSet\textbackslash{}Services}
	\item The Autoruns utility can be used to find potential persistence mechanisms. Task Scheduler and RunKeys should also be checked. \\
		\code{HKLM\textbackslash{}Software\textbackslash{}Microsoft\textbackslash{}Windows\textbackslash{}CurrentVersion\textbackslash{}Run} \\
		\code{HKLM\textbackslash{}Software\textbackslash{}Microsoft\textbackslash{}Windows\textbackslash{}CurrentVersion\textbackslash{}RunOnce} \\
		\code{HKCU\textbackslash{}Software\textbackslash{}Microsoft\textbackslash{}Windows\textbackslash{}CurrentVersion\textbackslash{}Run} \\
		\code{HKCU\textbackslash{}Software\textbackslash{}Microsoft\textbackslash{}Windows\textbackslash{}CurrentVersion\textbackslash{}RunOnce}
	\item You can use BLUESPAWN to aid in hunting, \bf{BUT DO NOT RELY ON IT SOLELY.}
	\begin{enumerate}
		\item It can be obtained from \href{https://github.com/ION28/BLUESPAWN/releases/download/v0.5.1-alpha/BLUESPAWN-client-x64.exe}{here}.
		\item Basic usage is as follows:
		\begin{enumerate}
			\item BLUESPAWN-client-x64.exe --mitigate compares the system to a secure baseline.
			\item BLUESPAWN-client-x64.exe --hunt does a single-pass scan for malicious activity.
			\item BLUESPAWN-client-x64.exe --monitor is like hunt but alerts on new activity.
		\end{enumerate}
	\end{enumerate}
\end{enumerate}

\pagebreak

\subsection{Considerations for Active Directory}
\label{subsec:adcon}

The below ports are needed for Active Directory to operate:
{ \color{iris} \begin{verbatim}
	53 TCP/UDP: DNS
	88 TCP/UDP: Kerberos
	123 TCP: NTP
	135 TCP: NetBIOS
	138,139 TCP/UDP: File Replication
	389,636 TCP: LDAP & LDAPS
	445 TCP: SMB
	464 TCP: Kerberos password change
	3268,3269 TCP: Global Catalog LDAP & LDAPS
\end{verbatim} }
\bf{These should be allowed inbound and outbound on a DC and outbound on a DM.}

The following template might help with making rules:
{ \color{iris} \begin{verbatim}
	$ports = 53,88,123,135,138,139,389,636,445,464,3268,3269;
	foreach ($p in $ports) {New-NetFirewallRule -DisplayName "AD $p" `
	-[LOCALPORT/REMOTEPORT] $p -Protocol [TCP/UDP] `
	-Action Allow -Direction [INBOUND/OUTBOUND]
	}
\end{verbatim} }

\begin{enumerate}
	\item Many of the above steps can be done across multiple machines via Group Policy.
	\item The krbtgt password should be reset.
	\item The Domain Administrators group should have minimum membership.
	\item Kerberos authentication attempts should be monitored.
	\item You can force a reset of domain group policy with the below commands:
		{ \color{iris} \begin{verbatim}
		dcgpofix /target:both
		gpupdate /force
		\end{verbatim} }
\end{enumerate}

\pagebreak

\subsection{Installing the ELK Stack}

\begin{enumerate}
	\item Install the prerequisites	using manually or using scoop (see below). \\
	\ul{If using scoop, apps are installed in \$HOME/scoop/apps/<APPNAME>/current.}
		\begin{enumerate}
			\item Add the `extras` repository with \code{scoop bucket add extras}.
			\item Add the `java' repository with \code{scoop bucket add java}.
			\item Install elasticsearch and kibana with \code{scoop install elasticsearch kibana openjdk}.
			\item Notepad++ (notepadplusplus) is recommended for editing the yml files.
		\end{enumerate}
	\item Add the following lines to \code{<ELASTICSEARCH DIR>/config/elasticsearch.yml}:
		{ \color{iris} \begin{verbatim}
			xpack.security.enabled: true
			xpack.security.enrollment.enabled: true
			cluster.initial\_master\_nodes: ["elk"]
			http.host: [\_local\_, \_site\_]
		\end{verbatim} }
	\item Start elasticsearch in a Powershell prompt. It will take a moment to initialize.
	\item Reset passwords for the kibana\_system and elastic accounts.
		\begin{enumerate}
			\item \code{<ELASTICSEARCH DIR>/bin/elasticsearch-reset-password -u kibana-system}
			\item \code{<ELASTICSEARCH DIR>/bin/elasticsearch-reset-password -u elastic}
		\end{enumerate}
	\bf{SAVE THESE PASSWORDS SOMEWHERE.}
	\item Add the following lines to \code{<KIBANA DIR>/config/kibana.yml}:
		{ \color{iris} \begin{verbatim}
			server.host: "0.0.0.0"
			server.publicBaseUrl: "http://localhost"
			elasticsearch.hosts: ["http://localhost:9200"]
			elasticsearch.username: "kibana\_system"
			elasticsearch.password: "<GENERATED KIBANA-SYSTEM PASSWORD>"
			elasticsearch.ssl.verificationMode: none
		\end{verbatim} }
	\item Stop the elasticsearch executable and install it as a service with: \\
		\code{<ELASTICSEARCH DIR>/bin/elasticsearch-service.bat install}
	\item Start the elasticsearch service with \code{start-service elasticsearch-service-x64}
	\item Start kibana in a Powershell prompt. It will take a moment to initialize.
	\item You may now log into kibana at localhost:5601 with the elastic account.
\end{enumerate}

\subsection{Configuring Winlogbeat}

\begin{enumerate}
	\item You can obtain Winlogbeat from \href{https://artifacts.elastic.co/downloads/beats/winlogbeat/winlogbeat-8.12.0-windows-x86\_64.zip}{here}.
	\item Extract the Winlogbeat directory to C:\textbackslash{}Program Files.
	\item Edit the winlogbeat.yml file as needed for environment and log forwarding.
	\item Install Winlogbeat as a service with \code{.\textbackslash{}install-service-winlogbeat.ps1}
	\item Load the prebuilt Winlogbeat assets with \code{.\textbackslash{}winlogbeat.exe setup -e}.
	\item Start the Winlogbeat service with \code{start-service winlogbeat}
\end{enumerate}

\pagebreak

\section{Service Configuration}
\label{sec:services}

\subsection{MySQL}

\begin{enumerate}
	\item Change passwords for any non-scored users. \\
		\code{SELECT User, Host FROM mysql.user;} \\
		\code{ALTER USER '<USERNAME>'@<HOST> IDENTIFIED BY '<NEW PASSWORD>';}
	\item Drop unauthorized users. \\
		\code{SELECT User, Host FROM mysql.user;} \\
		\code{DROP USER '<USERNAME>'@<HOST>; }
	\item Prepare a database backup and store somewhere safe \bf{off the system}.
	\begin{enumerate}
		\item Backup: \code{mysqldump -u <USERNAME> -p --all-databses > <BACKUP PATH>}
		\item Restore: \code{mysql -u <USERNAME> -p < <BACKUP PATH>} \\
	\end{enumerate}
\end{enumerate}

\subsection{FTP}

\bf{ANONYMOUS ACCESS CANNOT BE DISABLED.}

\begin{enumerate}
	\item On Windows, you will interact with the IIS console to manage FTP.
	\item On Linux, you will interact with /etc/vsftpd.conf (probably).
	\item Manage anonymous permissions to only allow reads (and writes if needed).
	\item Restrict access to non-shared directories from anonymous users.
	\item Prevent executables from running in the shared directory.
\end{enumerate}

\subsection{SSH}

\begin{enumerate}
	\item Check the SSH config file for potential misconfigurations.
	\begin{enumerate}
		\item \code{PermitRootLogin} should be set to \code{no} or \code{prohibit-password}
		\item \code{UsePAM} should be set to \code{no}
		\item \code{PasswordAuthentication} should be set to \code{yes}
		\item \code{PermitEmptyPasswords} should be set to \code{no}
	\end{enumerate}
	\item You can tunnel a port over ssh with the following syntax: \\
		\code{ssh -L <LPORT>:<RHOST>:<RPORT> <USER>@<RHOST>}
\end{enumerate}

\subsection{Web Servers}

\begin{enumerate}
	\item With IIS, some hardening can be automated with \href{https://github.com/ufsitblue/blue/blob/main/dsu\_blue/windows/IIS.ps1}{this script}.
	\item (If applicable) Change passwords for user accounts on website.
	\item Ensure that directory listing is disabled.
	\item Check web server directory for php webshells.
	\begin{enumerate}
		\item Webshells are likely to have malicious phrases like exec in the php file.
		\item You can minimize the possibility of this attack by disabling the phrases outright.
		\item Run php --ini to list all of the config files currently loaded into php.
		\item Add the following lines to the end of each config file:
	\end{enumerate}
	{ \color{iris} \begin{verbatim}
		# the below needs to be on one line
		disable\_functions=exec,passthru,shell\_exec,system,proc\_open,
		popen,curl\_exec,curl\_multi\_exec,parse\_ini\_file,show\_source
		
		# disable file uploads if they aren't needed
		file\_uploads=off
		allow\_url\_fopen=off
		allow\_url\_include=off
		\end{verbatim} }		
\end{enumerate}


\end{document}
