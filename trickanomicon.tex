\documentclass[12pt,letterpaper]{article}
\usepackage[margin=1in]{geometry}
\usepackage[T1]{fontenc}

% Font w/ xelatex
\ifxetex
	\usepackage{fontspec}
	\setsansfont{Jost.ttf}[
	BoldFont = Jost-Bold.ttf ,
	ItalicFont = Jost-Italic.ttf ,
	BoldItalicFont = Jost-BoldItalic.ttf]
	\renewcommand{\familydefault}{\sfdefault}
	\setmonofont{FiraCode Nerd Font}
\else
	\renewcommand{\familydefault}{\sfdefault}
\fi

% Quotes
\usepackage[american]{babel}

% Rose Pine
\usepackage{xcolor}
\definecolor{fg}{HTML}{e0def4}
\definecolor{bg}{HTML}{232136}
\definecolor{c0}{HTML}{393552}
\definecolor{c1}{HTML}{eb6f92}
\definecolor{c2}{HTML}{9ccfd8}
\definecolor{c3}{HTML}{f6c177}
\definecolor{c4}{HTML}{3e8fb0}
\definecolor{c5}{HTML}{c4a7e7}
\definecolor{c6}{HTML}{ea9a97}
\definecolor{c7}{HTML}{e0def4}
\definecolor{c8}{HTML}{6e6a86}
\pagecolor{bg}
\color{fg}

% Images
% \usepackage{graphicx}
% \usepackage{wrapfig}
% \usepackage{float}

% \usepackage{pgfplots}
% \usepackage{multicol}
% \usepackage{lipsum}

\def\code#1{\textcolor{c2}{\texttt{#1}}}
\def\bf#1{\textbf{#1}}
\def\ul#1{\underline{#1}}
\def\it#1{\textit{#1}}

% Links
\usepackage{hyperref}
\hypersetup{
    colorlinks=true,
    linkcolor=c4,
    filecolor=magenta,
    urlcolor=blue,
}

\title{\huge{\textit{Trickanomicon}}}
\author{Clemson CCDC Team}
\date{2024}

\begin{document}

\maketitle

\pagebreak

\tableofcontents

\pagebreak

\section{Linux}

\bf{Do not touch the} \code{seccdc\_black} \bf{account}

\subsection{30 Minute Plan}

\begin{enumerate}
	\item Start NMAP scan from your local computer to your assigned machine in the background
		\begin{enumerate}
			\item Do a quick scan to find all open ports \\
				\code{\bf{nmap -p- -T5} \ul{<target ip>}}
			\item After getting all open ports, get service and OS information with specific ports \\
				\code{\bf{nmap -sV -O -p port1,port2,port3,...} \ul{<target ip>}}
		\end{enumerate}
	\item Rotate all ssh keys
		\begin{enumerate}
			\item Populate the machines with the ssh key from Adrian/Duncan
			\item Deploy with \code{\bf{ssh-copy-id -i} \ul{identity\_file} [\ul{user}@\ul{hostname}]}
			\item Ensure there is only one entry (one line) in \code{/home/<user>/.ssh/authorized\_keys}. If there are more than one, remove all lines EXCEPT the very last one. After saving, make sure you can still SSH by trying on a NEW terminal.
			\item Remove all other authorized keys files with \code{find / -name authorized\_keys}
		\end{enumerate}
	\item Check accounts and reset password \bf{ASAP} using appropriate one liners from \bf{Duncan's Magic}
	\item Lock unnecessary accounts with \code{\bf{usermod -L} \ul{LOGIN}} and if nothing goes red, delete account with \code{\bf{userdel -r} \ul{LOGIN}}. Or use appropriate one liners from \bf{Duncan's Magic} to lock accounts
	\item Find listening services with \code{ss -lp} and investigate strange ones
\end{enumerate}

\subsection{Monitoring}

\begin{enumerate}
	\item View all network connections \code{ss -tunlp}
	\item View listening programs with \code{ss -lp}
	\item View only connections with \code{ss -tu}
	\item View active processes \code{ps -e}
	\item Continuously see processes with \code{top}
		\begin{enumerate}
			\item Sort by different categories with \code{<} and \code{>}
			\item Tree view with \code{V}
		\end{enumerate}
	\item Watch \bf{all} network traffic with the following (this is a fire hose)
		\begin{enumerate}
			\item Record traffic with \code{iptables -I INPUT -j LOG}
			\item Watch logs with \code{journalctl -kf --grep="IN=.*"}
		\end{enumerate}
	\item Watch dbus with \code{dbus -w}
\end{enumerate}

\subsection{System Utilities}

\begin{enumerate}
	\item Start and stop processes with \code{systemctl}
	\begin{enumerate}
		\item Start service with \code{\bf{systemctl start} [\ul{UNIT}]}
		\item Stop service with \code{\bf{systemctl stop} [\ul{UNIT}]}
		\item Restart service with \code{\bf{systemctl restart} [\ul{UNIT}]}
		\item Enable service with \code{\bf{systemctl enable} [\ul{UNIT}]}
	\end{enumerate}
	\item Permit and allow network connections with \code{iptables}
	\begin{enumerate}
		\item Default deny with \code{iptables -P INPUT DROP}
		\item Allow port access (must choose tcp or udp) \\
			\code{\bf{iptables -A INPUT -p tcp|udp --dport} \ul{port} \bf{-j ACCEPT}}
		\item Allow all from interface \\
			\code{\bf{iptables -A INPUT -i} \ul{interface} \bf{-j ACCEPT}}
	\end{enumerate}
	\item Schedule tasks with cron
	\begin{enumerate}
		\item View with \code{crontab -e}
		\item Obliterate user's crontab with \code{crontab -r}
	\end{enumerate}
\end{enumerate}

\subsection{Hunting}

\begin{enumerate}
	\item List all files with creation date, most recent first: \code{find /usr /bin /etc \textbackslash{} \\
		/var -type f -exec stat -c "\%W \%n" \{\} + | sort -r > files}
	\item List all files created after set date, most recent first: \code{find /usr /bin /etc \textbackslash{} \\
		/var -type f -newermt \ul{YYYY-MM-DD} -exec stat -c "\%W \%n" \{\} + | sort -r > files}
\end{enumerate}

\subsection{Duncan's Magic}

\begin{enumerate}
	\item Remove users listed in disable.txt \\
		\code{while read user;do sudo usermod -L \$user;done<disable.txt}
	\item Generate new passwords for every user in ./users.txt. Output format and filename as specified by SECCDC-2024 password reset guidelines. Ensure SERVICE is changed to match the corresponding service the passwords are being reset for. \\
		\code{t='25'; s='SERVICE'; while read u; do u=`echo \$u | tr -d '[:space:]'`; \textbackslash{}\\
		p=`tr -dc 'A-Za-z0-9!@\#\$\%' </dev/urandom | head -c 24; echo`; \textbackslash{}\\
		echo "\$s,\$u,\$p" >{}> "Team"\$t"\_"\$s"\_PWD.csv"; done < users.txt}
	\item Actually reset passwords using the generated list from the command immediately preceeding this \\
		\code{awk -F, '\{ print \$2":"\$3 \}' ./<generated\_file.csv> | sudo chpasswd}
	\item Find users not in ./known\_users.txt \\
		\code{awk -F: '\{if(\$3>=1000\&\&\$3<60000)\{print \$1\}\}' /etc/passwd|sort|comm -13 \textbackslash{} \\
		known\_users.txt ->extra\_users.txt}
\end{enumerate}

\end{document}
